\documentclass{article}
\usepackage{amsthm}

\newtheorem{remark}{Remark}

\title{Introduction to Game Theory}
\date{}

\begin{document}
\maketitle

\section{Introduction}

\subsection{What is game theory?}
Game theory (GT) is the field that explores strategic decision making. \\

\textbf{Not GT:}
\begin{itemize}
    \item How to play Roulette
    \item Whether to buy a given flat
\end{itemize}

\textbf{GT:}
\begin{itemize}
    \item How to play poker
    \item How to bargain the price of a flat you want to buy
\end{itemize}

\subsection{Elements of game theory.}

\begin{enumerate}
<<<<<<< HEAD
    \item Who are the players?
    \item What are the plaers' possible actions?
=======
    \item Who are the player?
    \item What are the players' possible actions?
>>>>>>> 7d4b02b90ced50c454b91bd18f823ee179787c6c
    \item Which informattion do players have when making their moves?
    \item What is the lemporal order of moves?
    \item What are the payoffs (ie what are the consequences of decisions)?
\end{enumerate}

In these notes we will formally define the elements of GT, and present examples.

\subsection{Who are we.}

The notes have been put together a team of GT enthusiasts, namely:

\begin{itemize}
    \item [!] Christian Hilbe
    \item [!] Charlotte Roseti
    \item [!] Marta Courto
    \item [!] Nikoleta Glynatsi
    \item [!] Saptarshi Pal
\end{itemize}

\section{Definitions \& Examples}

\subsection{Definition of a Normal Form Game}

Normal form games are also known as static games with complete information (SCGI). To define any game one needs the following - 1) set of players 2) their defined set of actions 3) payoff outcome mappings 4) order of moves 5) information availability. For Normal Form Games or SCGI the assumptions are on the the order of moves and information availability. It is assume all players move 1) simultaneously (static) and all players have 2) all relevant payoff related information (complete information). 


\subsection{Example: Matching Pennies}





\subsection{Example: Prisoner's Dilemma}





\subsection{Example: Hawk Dove}





\end{document}
