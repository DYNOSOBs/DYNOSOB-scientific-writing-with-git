\documentclass{article}
\usepackage{amsthm}

\newtheorem{remark}{Remark}

\title{Introduction to Game Theory}
\date{}

\begin{document}
\maketitle

\section{Introduction}

\subsection{What is game theory?}
Game theory (GT) is the field that explores strategic decision making. \\

\textbf{Not GT:}
\begin{itemize}
    \item How to play Roulette
    \item Whether to buy a given flat
\end{itemize}

\textbf{GT:}
\begin{itemize}
    \item How to play poker
    \item How to bargain the price of a flat you want to buy
\end{itemize}

\subsection{Elements of game theory.}

\begin{enumerate}
    \item Who are the player?
    \item What are the plaers' possible actions?
    \item Which informattion do players have when making their moves?
    \item What is the temporal order of moves?
    \item What are the payoff (ie what are the consequences of decisions)?
\end{enumerate}

In these notes we will formally define the elements of GT, and present examples.

\subsection{Who are we.}

The notes have been put together a team of GT enthusiasts, namely:

\begin{itemize}
    \item [!] Christian Hilde
    \item [!] Charlotte Roseti
    \item [!] Marta Couto
    \item [!] Nikoletta Glynatsi
    \item [!] Saptarshi Pali
\end{itemize}

\section{Definitions \& Examples}

\subsection{Definition of a Normal Form Game}





\subsection{Example: Matching Pennies}





\subsection{Example: Prisoner's Dilemma}

The Prisoner's dilemma is most studied social dilemma. You'd better mutually cooperate, but you don't wanna be the sucker either!

\subsection{Example: Hawk Dove}





\end{document}